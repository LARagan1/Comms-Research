\documentclass{article}
\usepackage{graphicx}
\usepackage{amsmath}
\usepackage{hyperref}
\hypersetup{
    colorlinks = true,
    linkcolor = blue,
    filecolor = magenta,
    urlcolor = cyan,
    pdftitle = {SDR Notes}
    pdfpagemode = FullScreen
}

\title{SDR Notes}
\author{Luke Ragan}
\date{March 2025}

\begin{document}

\maketitle

\section{Introduction}
Hello! My name is Luke Ragan. At the time of this writing, I am a senior 
electrical engineering student at Liberty University. I am writing these notes 
to document the approach that I have taken to set up Dr. Bae's software
defined radios (SDR). It is my hope that this guide will help future 
researchers to go further and achieve more. If at any point, you have 
questions about my work please feel free to send me an email at LARagan1@liberty.edu.
Without further ado, please enjoy!

\section{Setting up the Laptops}
The first step in this process has been to set up the Lenovo T410 Thinkpads that
belong to Dr. Bae. These machines have had Ubuntu 24.04.02 installed on them. 
This procedure was fairly straightforward, and one could easily repeat this by
result by navigating to ubuntu downloads and following the instructions. \\ \\
The next steps were to install GNU Radio, visual studio code, and latex on the laptops.
To install GNU Radio follow the instructions at the link below. \\
\indent \url{https://wiki.gnuradio.org/index.php?title=InstallingGR} \\
To install visual studio code, please use the app center included on the ubuntu machine. \\
To install LaTex, please follow the instructions at the link below. \\
\indent \url{https://github.com/James-Yu/LaTeX-Workshop/wiki/Install} \\
Please note, I used TinyTeX as the LaTex distribution on this laptop. \\
% Where to find information on how to use LaTex
Resorces for how to use LaTex can be found at the following addresses.
\begin{itemize}
    \item \url{https://www.overleaf.com/learn/latex/Learn_LaTeX_in_30_minutes}
    \item \url{https://www.learnlatex.org/en/}
    \item \url{https://ctan.math.illinois.edu/info/lshort/english/lshort.pdf}
\end{itemize}


\section{GNU Radio Tutorials}
I have begun my learning exercises for SDR by going through the 
tutorials for GNU Radio \href{https://wiki.gnuradio.org/index.php?title=Tutorials}{here}.
All of these tutorials are fairly self-explanatory. Additionally, the initial tutorials even
cover how to launch GNU radio companion.

\section{Ettus Research Tutorial}
In addition to the GNU Radio Tutorials, I also started a starter course that Ettus 
Research has published at \href{https://kb.ettus.com/images/4/47/Workshop_GnuRadio_Slides_20190507.pdf}{this site.}


\end{document}
