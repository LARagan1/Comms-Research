\documentclass{article}
\usepackage{graphicx}
\usepackage{amsmath}
\usepackage{hyperref}
\hypersetup{
    colorlinks = true,
    linkcolor = blue,
    filecolor = magenta,
    urlcolor = cyan,
    pdftitle = {SDR Notes}
    pdfpagemode = FullScreen
}

\title{SDR Notes}
\author{Luke Ragan}
\date{March 2025}

\begin{document}

\maketitle

\section{Introduction}
Hello! My name is Luke Ragan. At the time of this writing, I am a senior 
electrical engineering student at Liberty University. I am writing these notes 
to document the approach that I have taken to set up Dr. Bae's software
defined radios (SDR). It is my hope that this guide will help future 
researchers to go further and achieve more. If at any point, you have 
questions about my work please feel free to send me an email at LARagan1@liberty.edu.
Without further ado, please enjoy!

\section{Setting up the Laptops}
The first step in this process has been to set up the Lenovo T410 Thinkpads that
belong to Dr. Bae. These machines have had Ubuntu 24.04.02 installed on them. 
This procedure was fairly straightforward, and one could easily repeat this by
result by navigating to ubuntu downloads and following the instructions. \\ \\
The next steps were to install GNU Radio, visual studio code, and latex on the laptops.
To install GNU Radio follow the instructions at the link below. \\
\indent \url{https://wiki.gnuradio.org/index.php?title=InstallingGR} \\
To install visual studio code, please use the app center included on the ubuntu machine. \\
To install LaTex, please follow the instructions at the link below. \\
\indent \url{https://github.com/James-Yu/LaTeX-Workshop/wiki/Install} \\
Please note, I used TinyTeX as the LaTex distribution on this laptop. \\
% Where to find information on how to use LaTex
Resorces for how to use LaTex can be found at the following addresses.
\begin{itemize}
    \item \url{https://www.overleaf.com/learn/latex/Learn_LaTeX_in_30_minutes}
    \item \url{https://www.learnlatex.org/en/}
    \item \url{https://ctan.math.illinois.edu/info/lshort/english/lshort.pdf}
\end{itemize}


\section{GNU Radio Tutorials}
I have begun my learning exercises for SDR by going through the 
tutorials for GNU Radio \href{https://wiki.gnuradio.org/index.php?title=Tutorials}{here}.
All of these tutorials are fairly self-explanatory. Additionally, the initial tutorials even
cover how to launch GNU radio companion.

\section{Learning SDR Basics}
I found a textbook from Analog Devices, that breaks down the necessary portions of communication systems and
signal processing that you need to know to effectively use the SDRs. I would greatly recommend taking
a look at this text and at least reading the chapters that you do not feel as strong on. Especially
if you are someone who has not yet taken communication systems or dsp, I would recommend reading the book.
It is a great tool for learning what you need without having to learn everything. The text can be 
found at \url{https://www.analog.com/media/en/training-seminars/design-handbooks/Software-Defined-Radio-for-Engineers-2018/SDR4Engineers.pdf}.

\section{github}
I also created a github for communications research at Liberty, but the only files that I ended up creating for
said github are these notes. That being said, the repository can be found at \url{https://github.com/LARagan1/Comms-Research/tree/main}.
If you would like to join, just send me an email, and I can give you permissions.

\section{Ettus Research Tutorial}
In addition to the GNU Radio Tutorials, I also started a starter course that Ettus 
Research has published at \href{https://kb.ettus.com/images/4/47/Workshop_GnuRadio_Slides_20190507.pdf}{this site.}
\\ \\
Note: On page 100, there are a couple things that should be corrected.
\begin{enumerate}
    \item python-mako should be python3-mako instead 
    \item python-docutils should be python3-docutils instead 
    \item The third line should be corrected to git clone https://github.com/EttusResearch/uhd.git 
    \item Many of the commands need sudo to work
    \item You should probably run "sudo apt upgrade" before the cmake process 
    \item Instead of using release\_003\_009\_005 as they suggested, I just made a new branch. I used sudo git checkout -b LARagan1/customBranch
\end{enumerate}
Notice that the ethernet connection of the laptop needs to be set to 192.168.10.1 with a net 
mask of 255.255.255.0 \\
As I am writing this, I have not done so since I need Wi-Fi connectivity.
\\ \\
3/14/25: I have established connectivity with the USRP! It is definitely necessary to set the ip address
to the afformentioned address. \\
Also note that running any of the example files should include ./ before the command 
\\ \\
When I attempted to download and build GNU radio according to the instructions in the Ettus Tutorial,
I had a lot of difficulty with making sure that all of the required packages were installed. 
There are a couple things I think would help. Use python3 instead of python for install commands.
For instance, sudo apt install python3-opengl instead of python-opengl. For packages that are 
not able to be found, I found that the easiest way to go about it was to just use google. One of 
note is that you should run sudo apt-get install libqt5gstreamer-dev \\
Overall, I made sure that after the cmake command, everything was enabled except for gr-soapy 
for gnuradio.
\\ \\
\subsection{Installing Packages:}
I am now going to work on setting up the second laptop ENGR2 with UHD and GNURadio. To be as 
helpful as possible, I will do my best to create a list of every package that needed to be installed
and the commands used to install them.
\begin{itemize}
    \item sudo apt-get install -y git
    \item sudo apt-get install -y libboost-all-dev libusb-1.0-0-dev python3-mako doxygen python3-docutils cmake build-essential
    \item sudo apt-get install curl
    \item sudo curl -O \url{http://launchpadlibrarian.net/648013231/libtinfo5_6.4-2_amd64.deb}
    \&\& sudo dpkg -i libtinfo5\_6.4-2\_amd64.deb
    \&\& sudo curl -O \url{http://launchpadlibrarian.net/648013227/libncurses5_6.4-2_amd64.deb}
    \&\& sudo dpkg -i libncurses5\_6.4-2\_amd64.deb
    \item sudo apt-get install libncurses-dev
    \item sudo apt-get install python3-ruamel.yaml
    \item sudo apt-get install dpdk
    \&\& sudo apt-get upgrade
    \&\& sudo apt-get install -y libdpdk-dev
    \item sudo apt-get install -y python3-gevent
    \item sudo apt-get install python3-pyudev
    \item sudo apt-get install -y python3-pyroute2
    \item Make sure you run sudo apt upgrade before any cmake
    \item I installed libuhd4.8.0 instead of libuhd003
    \item I installed tree to view the uhd images in tree format (like in the presentation)
    \item To run the executables in the examples, navigate to /usr/local/lib/uhd/examples and make sure to put ./ in front of the command
    \item I was able to sucessfully run rx_ascii_art_dft and tx_waveforms simultaneously to verify that the SDRs and sending and receiving correctly
    \item On page 145, I made the following changes
    \begin{itemize}
        \item libcppunit-1.13-0v5 was replaced with libcppunit-1.15.0
        \item Replace any "python" with "python3"
        %\item To install qt4, go to \url{https://download.qt.io/archive/qt/4.8/}. Download the file, extract it with tar, and run the configure executable.
        %\item run sudo apt-get upgrade and sudo apt-get update afterward
        %\item For the purpose of time, I stopped working on setting up the second laptop after this point. I had many issues with the installation of qt4. I know that I did it on the first laptop, so it is possible, but I cannot remember how now.
        \item Run sudo apt-get install libqt5gstreamer-dev for qt4
        \item Replace python-wxgtk3.0 with python3-wxgtk4.0
        \item There are several things that did not install. I am just moving on, and I will fix any problems that occur later on.
    \end{itemize}
    
    \item Checkout the most recent version of gnuradio, not the one they recommend in the tutorial.
    \item To make the cmake work correctly, I installed the following packages
    \begin{itemize}
        \item qjackctl
        \item portaudio19-dev
        \item python3-libiio
        \item libsndfile-dev
        \item libqwt-qt5-dev 
        \item libiio-dev
        \item libad9361-dev 
        \item qtcreator
        \item libqt5charts5-dev libqt5datavisualization5-dev libqt5gamepad5-dev libqt5gstreamer-dev libqt5networkauth5-dev libqt5opengl5-dev libqt5remoteobjects5-dev libqt5scxml5-dev libqt5sensors5-dev libqt5serialbus5-dev libqt5serialport5-dev libqt5svg5-dev libqt5texttospeech5-dev libqt5virtualkeyboard5-dev libqt5waylandclient5-dev libqt5waylandcompositor5-dev libqt5webkit5-dev libqt5webchannel5-dev libqt5websockets5-dev libqt5webview5-dev libqt5x11extras5-dev libqt5xmlpatterns5-dev
        \item libzmq3-dev 
    \end{itemize}
    \item Make sure that the only disabled component is gr-soapy before moving on from the cmake

\end{itemize}

\section{Concluding}
Unfortunately, I did not get nearly as far as I was hoping to with this project. I successfuly setup uhd
and GNU Radio on Dr. Bae's laptops, verfied that the SDRs can send and receive simple signals, and 
hopefully gathered a few helpful sources for future researchers to loook at. I hope that this serves
as a helpful foundation. I look forward to seeing what research you are able to do!
Thank you!

\end{document}
